% Rien d'autre à faire qu'afficher le titre
\begin{frame}
    \titlepage{}
\end{frame}


% La table des matières utilise ce que vous donnez aux commandes \section et 
% \subsection tout au long de la présentation.
\begin{frame}
    \frametitle{Plan de l'exposé} 
    \tableofcontents 
\end{frame}


%\begin{frame}{Explication du problème et lien avec la ville}{Un petit calcul}
%    \begin{itemize}
%        \item <1-> Un pixel est un élément de \([\![0;255]\!]^3\) il nécessite donc \(3\) bits pour être stocké.
%        \item <2-> Une image de haute définition (HD) est composée de \(1280\times720\) pixels.
%        \item <3-> Une caméra enregistre à environ 30 images par secondes.
%        \item <4-> Une caméra de surveillance qui enregistrerait pendant 1 journée produirait donc \(30\times24\times60\times60 = 2592000\) images.
%        \item <5-> On obtiendrait donc un fichier de  \(7^{12}\) bits soit \(7\U{To}\). 
%    
%    \end{itemize}
%\end{frame}

\begin{frame}{Explication du problème et lien avec la ville}{Un petit calcul}
    \begin{itemize}
        \item <1-> Un pixel = \textbf{3 octets}
        \item <2-> Une image en HD = \textbf{921 600 pixels}
        \item <3-> Une seconde de vidéo = \textbf{30 images}
        \item <4-> Un enregistrement de surveillance de une journée = \textbf{7 To} 
        \item <5-> La taille d'un disque dur moyen = \textbf{1 To}
    \end{itemize}
\end{frame}
