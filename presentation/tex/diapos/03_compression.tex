\section{Algorithmes de compression}
%%%%%%%%%%%%%%%%%%%%%%%%%%%%%%%%%%%%%%%%%%%%%%%%
% Première diapo
%%%%%%%%%%%%%%%%%%%%%%%%%%%%%%%%%%%%%%%%%%%%%%%%
\begin{frame}{Algorithmes de compression}{Méthodes de quantification}
   \onslide<1-> \begin{block}{Différentes méthodes envisagées :}
        \begin{itemize}
            \item <2-> Selection par seuil
            \item <3-> Selection par zone
            \item <4-> Matrice de Quantification
        \end{itemize}
    \end{block} 
\end{frame}

%%%%%%%%%%%%%%%%%%%%%%%%%%%%%%%%%%%%%%%%%%%%%%%%
%% Deuxième diapo
%%%%%%%%%%%%%%%%%%%%%%%%%%%%%%%%%%%%%%%%%%%%%%%%%
%\begin{frame}{Algorithme de compression}{Exemples : Selection par zone}
%    On garde uniquement les coefficients au dessus d'une certaines zones
%
%    \begin{block}{Selection par zone}
%        \scalebox{0.8}[1.5]{
%            \begin{tabular}{|c|c|c|c|c|c|c|c|}
%                \hline 1722 & -108 & 3 & -2 & -4 & 0 & 0 & -2  \\
%                \hline 76 & 12 & -25 & -1 & 0 & 0 & -2 & 1 \\ 
%                \hline 12 & -4 & 4 & -1 & -2 & -1 & 1 & 0 \\ 
%                \hline 0 & 1 & 2 & -2 & 1 & -1 & 0 & 1 \\ 
%                \hline -4 & -1 & 0 & 1 & 0 & 0 & 1 & 0 \\ 
%                \hline -3 & 6 & -1 & -1 & 0 & 0 & -1 & 1 \\ 
%                \hline 1 & -3 & 0 & 1 & 0 & 0 & 0 & -1 \\ 
%                \hline 0 & 1 & 0 & -1 & 0 & 0 & 0 & 1 \\\hline
%            \end{tabular}
%            \(\rightarrow\)
%            \begin{tabular}{|c|c|c|c|c|c|c|c|}
%                \hline 1722 & -108 & 3 & -2 & -4 & 0 & 0 & -2  \\
%                \hline 76 & 12 & -25 & -1 & 0 & 0 & -2 & 1 \\ 
%                \hline 12 & -4 & 4 & -1 & -2 & -1 & 1 & 0 \\ 
%                \hline 0 & 1 & 2 & 0 & 0 & 0 & 0 & 0 \\ 
%                \hline -4 & -1 & 0 & 0 & 0 & 0 & 0 & 0 \\ 
%                \hline -3 & 6 & -1 & 0 & 0 & 0 & 0 & 0 \\ 
%                \hline 1 & -3 & 0 & 0 & 0 & 0 & 0 & 0 \\ 
%                \hline 0 & 1 & 0 & 0 & 0 & 0 & 0 & 0 \\\hline
%            \end{tabular}
%        }
%    \end{block}
%\end{frame}
%    
%%%%%%%%%%%%%%%%%%%%%%%%%%%%%%%%%%%%%%%%%%%%%%%%%
%% Troisième diapo
%%%%%%%%%%%%%%%%%%%%%%%%%%%%%%%%%%%%%%%%%%%%%%%%%
%\begin{frame}{Algorithme de compression}{Exemples : Selection par seuil}
%    On garde uniquement les coefficients au dessus d'un certain seuil
%
%    Mettre des exemples
%    \begin{align*}
%        \onslide<1->{
%            \begin{pmatrix}
%                1 & 2 & 3 & 4 & 5 & 6 & 7 & 8 & 9 \\
%                1 & 2 & 3 & 4 & 5 & 6 & 7 & 8 & 9 \\
%                1 & 2 & 3 & 4 & 5 & 6 & 7 & 8 & 9 \\
%                1 & 2 & 3 & 4 & 5 & 6 & 7 & 8 & 9 \\
%                1 & 2 & 3 & 4 & 5 & 6 & 7 & 8 & 9 \\
%                1 & 2 & 3 & 4 & 5 & 6 & 7 & 8 & 9 \\
%                1 & 2 & 3 & 4 & 5 & 6 & 7 & 8 & 9 \\
%                1 & 2 & 3 & 4 & 5 & 6 & 7 & 8 & 9 \\
%            \end{pmatrix}
%        }
%        \onslide<2->{\longrightarrow}
%        \onslide<3->{
%            \begin{pmatrix}
%                1 & 2 & 3 & 4 & 5 & 6 & 7 & 8 & 9 \\
%                1 & 2 & 3 & 4 & 5 & 6 & 7 & 8 & 9 \\
%                1 & 2 & 3 & 4 & 5 & 6 & 7 & 8 & 9 \\
%                1 & 2 & 3 & 4 & 5 & 6 & 7 & 8 & 9 \\
%                1 & 2 & 3 & 4 & 5 & 6 & 7 & 8 & 9 \\
%                1 & 2 & 3 & 4 & 5 & 6 & 7 & 8 & 9 \\
%                1 & 2 & 3 & 4 & 5 & 6 & 7 & 8 & 9 \\
%                1 & 2 & 3 & 4 & 5 & 6 & 7 & 8 & 9 \\
%            \end{pmatrix}
%        }
%    \end{align*}
%\end{frame}
%
%

%%%%%%%%%%%%%%%%%%%%%%%%%%%%%%%%%%%%%%%%%%%%%%%%
% Quatrième diapo
%%%%%%%%%%%%%%%%%%%%%%%%%%%%%%%%%%%%%%%%%%%%%%%%
\begin{frame}{Algorithmes de compression}{Exemples : Matrice de Quantification}
    \begin{block}{Matrice de quantification}
        A partir d'une matrice de quantification Q.
        \newline\newline
        \onslide<2-> L'opération quantification est la suivante : \\
        \onslide<3-> \[\widehat{A}_{i,j} = \left\lfloor\frac{A_{i,j}}{Q_{i,j}}\right\rfloor\] \\

        \onslide<4-> La déquantification s'écrit de la manière suivante : \\
        \[A_{i,j} \approx \widehat{A}_{i,j}\times Q_{i,j} \]
    \end{block}
\end{frame}

%%%%%%%%%%%%%%%%%%%%%%%%%%%%%%%%%%%%%%%%%%%%%%%%
% Cinquième diapo
%%%%%%%%%%%%%%%%%%%%%%%%%%%%%%%%%%%%%%%%%%%%%%%%
%\begin{frame}{Algorithme de compression}{Exemples : Matrice de Quantification}
%
%\end{frame}

%%%%%%%%%%%%%%%%%%%%%%%%%%%%%%%%%%%%%%%%%%%%%%%%
% Sixième diapo
%%%%%%%%%%%%%%%%%%%%%%%%%%%%%%%%%%%%%%%%%%%%%%%%
\begin{frame}{Algorithmes de compression}{Algorithme formel}
    \onslide<1->
    \begin{block}{Algorithme de compression}
        \begin{itemize}
            \item <1-> Tranformation des couleurs en YCbCr
            \item <2-> Tranformée en Cosinus discrète
            \item <3-> Quantification des plans de Luminances et Chrominances
            \item <4-> Compression sans pertes
        \end{itemize}    
    \end{block}
    \onslide<5->
    \begin{block}{Algorithme de décompression}
        \begin{itemize}
            \item Decompression
            \item Déquantification
            \item Tranformée inverse
            \item Tranformation des couleurs en RGB
        \end{itemize}
    \end{block}
\end{frame}

%%%%%%%%%%%%%%%%%%%%%%%%%%%%%%%%%%%%%%%%%%%%%%%%
% Septième diapo
%%%%%%%%%%%%%%%%%%%%%%%%%%%%%%%%%%%%%%%%%%%%%%%%