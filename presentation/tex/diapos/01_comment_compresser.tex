\section{Comment compresser une image}

%%%%%%%%%%%%%%%%%%%%%%%%%%%%%%%%%%%%%%%%%%%%%%%%
% Première diapo
%%%%%%%%%%%%%%%%%%%%%%%%%%%%%%%%%%%%%%%%%%%%%%%%
\begin{frame}{Comment compresser une image}{Une première approche \dots}
    \begin{align*}
        \underbrace{111\dots111}_{\text{100 fois}} \longrightarrow \textrm{100*1} 
    \end{align*} 

    \begin{center}
        On gagne un facteur \textbf{20} !
    \end{center}
\end{frame}

%%%%%%%%%%%%%%%%%%%%%%%%%%%%%%%%%%%%%%%%%%%%%%%%
% Deuxième diapo
%%%%%%%%%%%%%%%%%%%%%%%%%%%%%%%%%%%%%%%%%%%%%%%%
\begin{frame}{Comment compresser une image}{\dots peu satisfaisante.}
    \begin{itemize}
        \item \(\displaystyle\sum_{k=0}^{N-x-1}2^k = 2^{N - x} - 1\) fichiers de taille strictement inférieur à \(N-x\) bits.
        \item \(2^N\) fichiers de taille $N$ bits.
    \end{itemize}
    On a une proportion de \(\frac{2^{N-10} - 1}{2^N} < \frac{1}{2^{10}} = \mathbf{\frac{1}{1024}}\) fichiers compressibles d'au moins 10 bits
\end{frame}

%%%%%%%%%%%%%%%%%%%%%%%%%%%%%%%%%%%%%%%%%%%%%%%%
% Troisième diapo
%%%%%%%%%%%%%%%%%%%%%%%%%%%%%%%%%%%%%%%%%%%%%%%%
%\begin{frame}{Comment compresser une image}{Compression avec pertes}
%    \begin{codi}
%        \obj{A &&&&&& B \\
%             C &&&&&& D\\};
%
%        \mor :[bend left] A transformation :-> B;
%        \mor : B quantification :-> C;
%        \mor :[bend right] C compression :-> D;
%    \end{codi}
%\end{frame}

\begin{frame}{Comment compresser une image}{Compression avec pertes}
    \begin{figure}[!ht]
        \centering
        \resizebox{0.7\textwidth}{!}{%
        \begin{circuitikz}
        \tikzstyle{every node}=[font=\Huge, line width = 40]
        \draw[line width = 3.0] (0,18) rectangle  node {Image} (5,12);
        \draw[line width = 3.0, -Stealth] (5,15) -- (7,15);
        \draw[line width = 3.0] (7,18) rectangle  node[align = center] {Image\\transformée} (12,12);
        \draw[line width = 3.0, -Stealth] (12,15) -- (14,15);
        \draw[line width = 3.0] (14,18) rectangle  node[align = center] {Image\\quantifiée} (19,12);
        \draw[line width = 3.0, -Stealth] (19,15) -- (21,15);
        \draw[line width = 3.0] (21,18) rectangle  node[align = center] {Image\\compressée} (26,12);
        \end{circuitikz}
        }%
        \caption{Schéma d'un algorithme de compression d'image}
    \end{figure}
\end{frame}