\section{Annexe}

\begin{frame}{Annexe}{Transformation de Fourier inverse}
    \begin{block}{Avec séparabilité}
        \[M = A^{-1}\widehat{M}(A^{-1})^t, \textrm{où} \; A^{-1} = {\left(\frac{1}{\sqrt{n}}\exp\left(-2\mathbf{i}\pi\frac{kl}{n}\right)\right)}_{\substack{0 \leqslant k \leqslant n-1,\\ 0 \leqslant l \leqslant n-1}}\]
    \end{block}
\end{frame}

\begin{frame}{Annexe}{Preuve que la matrice de la transformée de Fourier est unitaire}
    \footnotesize{
        \[{\textrm{On note : } A = \left(\frac{1}{\sqrt{n}}\exp\left(2\mathbf{i}\pi\frac{kl}{n}\right)\right)}_{\substack{0 \leqslant k \leqslant n-1,\\ 0 \leqslant l \leqslant n-1}}
        B = {\left(\frac{1}{\sqrt{n}}\exp\left(-2\mathbf{i}\pi\frac{kl}{n}\right)\right)}_{\substack{0 \leqslant k \leqslant n-1,\\ 0 \leqslant l \leqslant n-1}}\]

        Il suffit de montrer que \([AB]_{i,j} = \delta_{i,j} \;\forall (i,j) \in [\![0;n-1]\!]^2 \)

        Soit \((i,j) \in [\![0;n-1]\!]^2 \)
        \begin{align*}
            [AB]_{i,j} &= \sum_{k=0}^{n-1} \frac{1}{\sqrt{n}}\exp\left(2\mathbf{i}\pi\frac{ki}{n}\right)\frac{1}{\sqrt{n}}\exp\left(2\mathbf{i}\pi\frac{kj}{n}\right)\\
            &= \frac{1}{n}\sum_{k=0}^{n-1}\left(\exp\left(2\mathbf{i}\pi\frac{i-j}{n}\right)\right)^k
        \end{align*}

        
        \[\textrm{Si } i = j\; [AB]_{i,j} = \frac{1}{n}\sum_{k=0}^{n-1} \exp{0} = 1 \;\;
        \textrm{Sinon } [AB]_{i,j} = \frac{1}{n}\frac{\exp\left(2\mathbf{i}\pi\frac{i-j}{n}\right)^n - 1}{\exp\left(2\mathbf{i}\pi\frac{i-j}{n}\right) - 1} = 0\]

        
        \[\textrm{On a bien } AB = I_n \textrm{, finalement } A \textrm{ est inversible d'inverse }  A^{-1} = {\left(\frac{1}{\sqrt{n}}\exp\left(-2\mathbf{i}\pi\frac{kl}{n}\right)\right)}_{\substack{0 \leqslant k \leqslant n-1,\\ 0 \leqslant l \leqslant n-1}}\]
    }
\end{frame}

\begin{frame}{Annexe}{Algorithme de la TCD}
    \begin{block}{Avec séparabilité}
        \[\widehat{M} = BMB^t, \textrm{où} \; B = {\left(\frac{\sqrt{2}C(l)}{\sqrt{n}}\cos\left(\frac{(2l+1)\pi k}{2n}\right)\right)}_{\substack{0 \leqslant k \leqslant n-1,\\ 0 \leqslant l \leqslant n-1}}\]
        \raggedleft{\(\textrm{où}\; C(0) = \frac{1}{\sqrt{2}}; C(k) = 1 \textrm{ pour } k \ne 0\)}

        \center L'inverse de \(M\) est sa transposée car elle est orthogonale.
    \end{block}
\end{frame}

\begin{frame}{Annexe}{Preuve que la matrice de la transformée en cosinus discrète est orthogonale}
    \footnotesize{
    Il suffit de montrer que les vecteurs colone de \(B\) sont orthogonaux.

    \[\textrm{On note } B_i = \left(\frac{\sqrt{2}C(l)}{\sqrt{n}}\cos\left(\frac{(2k+1)i\pi}{2n}\right)\right)_{0 \leqslant k \leqslant n - 1} \textrm{ le } i\textrm{eme vecteur colone de B}\]

    Montrons que \(<B_i, B_j> \;= \delta_{i,j} \;\; \forall (i,j) \in [\![0;n-1]\!]^2 \)
    
    \begin{align*}
        <B_i, B_j> &= \frac{2C(i)C(j)}{n}\sum_{k = 0}^{n-1}\cos\left(\frac{(2k+1)i\pi}{2n}\right)\cos\left(\frac{(2k+1)j\pi}{2n}\right)\\
        &= \frac{C(i)C(j)}{n}\sum_{k = 0}^{n-1}\cos\left(\frac{(2k+1)(i+j)\pi}{2n}\right)+\cos\left(\frac{(2k+1)(i-j)\pi}{2n}\right)
    \end{align*}

    }
\end{frame}
\begin{frame}{Annexe}{Preuve que la matrice de la transformée en cosinus discrète est orthogonale}
    \footnotesize{

    \begin{align*}
        \sum_{k = 0}^{n-1}\cos\left(\frac{(2k+1)(i+j)\pi}{2n}\right) &= \mbox{Re}\!\left[\sum_{k = 0}^{n-1}\exp\left(\mathbf{i}\frac{(2k+1)(i+j)\pi}{2n}\right)\right]\\
        &= \mbox{Re}\!\left[ \exp\left(\mathbf{i}\frac{(i+j)\pi}{2n}\right) \sum_{k = 0}^{n-1}\left(\exp\left(\mathbf{i}\frac{(i+j)\pi}{n}\right)\right)^k\right]\\
        &= \mbox{Re}\!\left[ \exp\left(\mathbf{i}\frac{(i+j)\pi}{2n}\right) \frac{(-1)^{i+j} - 1}{\exp\left(\mathbf{i}\frac{(i+j)\pi}{n}\right) - 1}\right]\\
    \end{align*}

    \[\textrm{Si } i+j \textrm{ est pair alors}  \sum_{k = 0}^{n-1}\cos\left(\frac{(2k+1)(i+j)\pi}{2n}\right) = 0\]
   
    }
\end{frame}
\begin{frame}{Annexe}{Preuve que la matrice de la transformée en cosinus discrète est orthogonale}
    \tiny{

    \begin{align*}
        \textrm{Sinon } \sum_{k = 0}^{n-1}\cos\left(\frac{(2k+1)(i+j)\pi}{2n}\right) &= \mbox{Re}\!\left[ \exp\left(\mathbf{i}\frac{(i+j)\pi}{2n}\right) \frac{-2}{\exp\left(\mathbf{i}\frac{(i+j)\pi}{n}\right) - 1}\right]\\
        &= \mbox{Re}\!\left[ \exp\left(\mathbf{i}\frac{(i+j)\pi}{2n}\right) \frac{-2\left(\exp\left(\mathbf{-i}\frac{(i+j)\pi}{n}\right) - 1\right)}{\left(\exp\left(\mathbf{i}\frac{(i+j)\pi}{n}\right) - 1\right)\left(\exp\left(\mathbf{i}\frac{(i+j)\pi}{n}\right) - 1\right)}\right]\\
        &= \mbox{Re}\!\left[ \frac{4\mathbf{i}\sin\left(\frac{(i+j)\pi}{2n}\right)}{2 - 2\cos\left(\frac{(i+j)\pi}{n}\right)}\right]\\
        &= 0
    \end{align*}

    \[\textrm{De même } \sum_{k = 0}^{n-1}\cos\left(\frac{(2k+1)(i-j)\pi}{2n}\right) = 0 \;\; \textrm{(Calculs similaires)}\] 

    \[\textrm{Finalement} <B_i,B_j> \;= 0 \;\forall (i,j) \in [\![0;n-1]\!]^2\]

    }
\end{frame}
\begin{frame}{Annexe}{Preuve que la matrice de la transformée en cosinus discrète est orthogonale}
    \footnotesize{
    \begin{align*}
        <B_i,B_j> \;&= \frac{2C(i)^2}{n}\sum_{k = 0}^{n-1}\cos^2\left(\frac{(2k+1)i\pi}{2n}\right)\\
        &= \frac{C(i)^2}{n}\sum_{k = 0}^{n-1}1+\cos\left(\frac{(2k+1)i\pi}{n}\right)\\
        &= 1 \;\;\left(\textrm{la somme vaut } 2n \textrm{ si } i = 0 \textrm{ et } C(0)^2 = \frac{1}{2}\right)
    \end{align*}
    }
\end{frame}


\begin{frame}{Annexe}{Matrices de quantification}
    \begin{figure}
        \begin{align*}
            \begin{pmatrix}
                16 & 11 & 10 & 16 & 24  & 40 & 51 & 61 \\
                12 & 12 & 14 & 19 & 26  & 58 & 60 & 55 \\
                14 & 13 & 16 & 24 & 40  & 57 & 69 & 56 \\
                14 & 17 & 22 & 29 & 51  & 87 & 80 & 62 \\
                18 & 22 & 37 & 56 & 68  & 109 & 103 & 77 \\
                24 & 35 & 55 & 64 & 81  & 104 & 113 & 92 \\
                49 & 64 & 78 & 87 & 103 & 121 & 120 & 101 \\
                72 & 92 & 95 & 98 & 112 & 100 & 103 & 99 \\
            \end{pmatrix}
        \end{align*}
        \caption{Matrice 1}    
    \end{figure}
\end{frame}

\begin{frame}{Annexe}{Matrices de quantification}
    \begin{figure}
        \begin{align*}
            \begin{pmatrix}
                5 & 9 & 13 & 17 & 21 & 25 & 29 & 33 \\
                9 & 13 & 17 & 21 & 25 & 29 & 33 &37 \\
                13 & 17 & 21 & 25 & 29 & 33 & 37 &41 \\
                17 & 21 & 25 & 29 & 33 & 37 & 41 &45 \\
                21 & 25 & 29 & 33 & 37 & 41 & 45 &49 \\
                25 & 29 & 33 & 37 & 41 & 45 & 49 &53 \\
                29 & 33 & 37 & 41 & 45 & 49 & 53 &57 \\
                33 & 37 & 41 & 45 & 49 & 53 & 57 &61 \\
            \end{pmatrix}
        \end{align*}
        \caption{Matrice 2}
    \end{figure}
\end{frame}

\begin{frame}{Annexe}{Matrices de quantification}
    \begin{figure}
        \begin{align*}
            \begin{pmatrix}
                3 & 5 & 7 & 9 & 11 & 13 & 15 & 17 \\
                5 & 7 & 9 & 11 & 13 & 15 & 17 & 19 \\
                7 & 9 & 11 & 13 & 15 & 17 & 19 & 21 \\
                9 & 11 & 13 & 15 & 17 & 19 & 21 & 23 \\
                11 & 13 & 15 & 17 & 19 & 21 & 23 & 25 \\
                13 & 15 & 17 & 19 & 21 & 23 & 25 & 27 \\
                15 & 17 & 19 & 21 & 23 & 25 & 27 & 29 \\
                17 & 19 & 21 & 23 & 25 & 27 & 29 & 31 \\ 
            \end{pmatrix}
        \end{align*}    
        \caption{Matrice 3}
    \end{figure}
\end{frame}